\chapter{Lista de Abreviações}

% exemplos de uso dos comandos de abreviaturas:

% \ac{} ou \acf{} abreviatura completa junto com a sigla
% 	ex: \ac{USP} -> Universidade de São Paulo (USP)
% \acfp{} abreviatura completa junto com a sigla, ambos no plural 
% 	ex: \acfp -> Universidade de São Paulos (USPS)
% \acl{} abreviatura completa sem a sigla
% 	ex: \acl{USP} -> Universidade de São Paulo
% \acp{} abreviatura completa com a sigla no plural
% 	ex: \acp{USP} -> Universidade de São Paulo (USPs)
% \acs{} apenas a sigla
% 	ex: \acs{USP} -> USP

\begin{acronym}
    \acro{API}{\textit{Application Programming Interface}}
    \acro{HTTP}{\textit{Hypertext Transfer Protocol}}
    \acro{IA}{Inteligência Artificial}
    \acro{RDBMS}{\textit{Relational Database Management System}}
    \acro{REST}{\textit{Representational State Transfer}}
    \acro{UI}{\textit{User Interface}}
    \acro{USP}{Universidade de São Paulo}
    \acro{UX}{\textit{User Experience}}
    \acro{URL}{\textit{Uniform Resource Locator}}
\end{acronym}
