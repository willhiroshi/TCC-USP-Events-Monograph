%!TeX root=../tese.tex
%("dica" para o editor de texto: este arquivo é parte de um documento maior)
% para saber mais: https://tex.stackexchange.com/q/78101

\chapter{Tecnologias}

O intuito inicial do projeto era criar uma plataforma que pudesse ser
facilmente acessada através de navegadores que todos estão acostumados a usar
no dia-a-dia, dessa forma, tornou-se necessário a definição de como poderia ser
estruturado o projeto e quais as ferramentas a serem utilizadas a fim de
viabilizar a criação de uma plataforma \textit{web} que atendessem ao objetivo
inicial do projeto.

Em relação à estrutura do projeto, a princípio, a separação mais clara e óbvia
foi a escolhida. O sistema foi dividido em dois sub projetos, \textit{back-end}
o qual ficaria responsável pelo armazenamento de dados, obtenção dos textos
fonte e processamento das informações; e \textit{front-end} que seria utilizado
para exibição dos dados e interação do usuário com a plataforma.

Ao longo do desenvolvimento foi se tornando cada vez mais clara uma nova
separação para a estrutura geral, assim, um novo módulo chamado de
\textit{scripts} foi criado no projeto, onde foram alocados arquivos que não se
encaixavam exatamente no escopo que estava sendo definindo para o
\textit{back-end}.

Dessa forma, a aplicação está construída em três partes principais:
\textit{front-end}, \textit{back-end} e \textit{webscraper scripts}. A seguir,
será descrita em mais detalhes a escolha das tecnologias utilizadas e o
objetivo de cada camada.

\section{\textit{Back-end}}

O \textit{back-end} é uma parte crucial de um sistema \textit{web} responsável
pelo processamento, gerenciamento, armazenamento e segurança de dados. Ele age
como uma camada intermediária que permite aos usuários interagirem com o
sistema sem a necessidade de entender as complexidades de suas operações.

A comunicação com o \textit{front-end} foi feita implementando a arquitetura
\ac{REST} e \acp{API}. Essa abordagem permite ao \textit{front-end} solicitar
dados e realizar ações no \textit{back-end} de maneira eficiente e padronizada,
tornando-se amplamente utilizada em sistemas \textit{web}.

No projeto, optou-se pelo \textit{framework} Django como ponto de partida para
a criação da \acs{API}.

\subsection{Django}

Uma das partes principais do projeto é o conjunto de \textit{scripts} para
automação da coleta e processamento de dados de redes sociais escritos em
Python. Já que o Django é também baseado nessa linguagem, isso permite uma
interação direta e simples entre ambas as camadas (\textit{back-end} e
\textit{scripts}).

Dado que nunca tivemos contato com o Django, um fator decisivo para sua escolha
é a sua grande comunidade de usuários, o que proporciona muito conhecimento
online, uma documentação detalhada e pronta ajuda em eventuais problemas e
dúvidas. Ainda mais, ele possui código aberto, alinhando-se perfeitamente com a
natureza acadêmica do projeto.

Outro ponto que acelerou significativamente o desenvolvimento foi a vasta
quantidade de bibliotecas disponíveis para o Django. A autenticação de usuários
dentro da plataforma é um dos exemplos que foi facilitada pelo uso de
bibliotecas.

\subsection{SQLite}

Dado que o sistema necessita gerenciar dados sobre essencialmente usuários e
possíveis eventos futuros, estava claro o uso de um banco de dados relacional
desde o início. Essas entidades possuem um formato predefinido e podem ser
facilmente conectados em um modelo de dados relacional, portanto restava apenas
decidir qual programa utilizar para manejar a informação.

O SQLite é um \ac{RDBMS} de código aberto responsável por criar, atualizar e
gerenciar bancos de dados relacionais. Uma de suas principais características é
o armazenamento de informações em arquivos locais, dispensando o uso de um
servidor de banco de dados separado. Essa propriedade torna o SQLite adequado
para projetos de menor escala como o construído, uma vez que não exige a
complexidade de um processo diferente com o servidor em execução, como o
PostgreSQL ou MySQL.

Outro ponto de interesse é sua capacidade de manejar baixo a médio tráfego de
requisições \acs{HTTP}\label{acro:HTTP}, o que é ideal para o caso de uso, já
que inicialmente, não existirão muitos usuários na plataforma.

Além disso, o SQLite já é integrado nativamente ao \textit{framework} Django.
Isso permitiu que o desenvolvimento fosse mais ágil, sem necessidade de
configurações adicionais para ter acesso ao banco de dados.

\section{\textit{Front-end}}

O \textit{front-end} é igualmente importante em um sistema \textit{web},
responsável pela interface gráfica disponibilizada ao usuário final. Dois
aspectos englobados pelo \textit{front-end} são: experiência do usuário, a
apresentação de informações e a interatividade, conjunto denominado de \ac{UX};
e a experiência visual do sistema voltado a estética da plataforma, área
denominada de \ac{UI}.

Existem inúmeros \textit{frameworks} que oferecem ferramentas para o
desenvolvimento dessas camadas do \textit{front-end} de maneira mais simples e
ágil. Alguns deles são Angular, Svelte, Vue.JS e React.JS. A seguir, serão
listados alguns motivos pelos quais decidiu-se pelo uso deste último.

\subsection{React.JS}

O principal motivo da escolha do React.JS para a construção do
\textit{front-end} foi a sua enorme popularidade, o que assegurou a existência
de muito conteúdo \textit{online} e auxílio da comunidade disponível. Isso
também indiretamente deu acesso a muitas bibliotecas externas como o
\textit{Material-UI} que oferece componentes de \acs{UI} prontos para uso,
permitindo o foco nas funcionalidades do projeto.

Além disso, ambos os integrantes do time já trabalharam com React.JS em
projetos passados dentro da universidade, como na matéria de MAC0472
(Laboratório de Métodos Ágeis). Essa oportunidade proporcionou uma base robusta
e confortável para a utilização deste mesmo \textit{framework} neste trabalho,
reduzindo a curva de aprendizado e aumentando a qualidade geral do projeto.

\section{\textit{Webscraping}}

A plataforma necessita de informações de possíveis eventos oriundas de fontes
externas. A coleta desses dados é automatizada utilizando \textit{scripts}
escritos em Python e realizada em cima de redes sociais usando ferramentas como
Selenium e BeautifulSoup, essa técnica para adquirir os dados é denominado de
\textit{Webscraping} ou raspagem de dados.

\subsection{Selenium e BeautifulSoup}

Existem várias bibliotecas mantidas em Python específicas para raspagem de
informações de redes sociais, facilitando bastante o processo. Porém, é comum o
bloqueio da utilização dessas ferramentas, por conta de possíveis violações dos
termos de uso de cada plataforma. Por esse motivo, optou-se pelo uso do
Selenium, comumente empregado em testes automatizados do \textit{front-end},
mas podendo ser aplicado também para \textit{Webscraping}.

Selenium é uma ferramenta capaz de automatizar funções do navegador,
possibilitando interagir com \textit{websites} como se fossem usuários comuns,
realizando ações como cliques e deslize de tela. Por conta dessa natureza, o
problema de suspensão durante a navegação \textit{web} é menos frequente e
oferece mais flexibilidade ao procurar informações nas redes sociais.

Enquanto o Selenium busca o conteúdo, o BeautifulSoup é responsável por
efetivamente extrair as informações. Ele é uma biblioteca capaz de obter dados
de arquivos HTML, permitindo adquirir o texto de \textit{tags} HTML
específicas, sendo exatamente o caso de uso do sistema para obter o conteúdo
das postagens das mídias sociais.