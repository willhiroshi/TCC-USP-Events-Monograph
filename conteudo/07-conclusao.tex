%!TeX root=../tese.tex
%("dica" para o editor de texto: este arquivo é parte de um documento maior)
% para saber mais: https://tex.stackexchange.com/q/78101

\chapter{Conclusão}

Ao final deste projeto, foi desenvolvida uma plataforma capaz de compilar
informações a respeito de eventos de caráter cultural e esportivo que vão
ocorrer dentro do ambiente universitário da USP, além de permitir a
possibilidade de personalização com base no interesse dos usuários para que
sejam exibidas apenas informações relevantes e de preferência do mesmo. A
plataforma pode ser acessada através do endereço:
\url{https://uspevents.ix.tc/}.

Vale destacar que a partir do ideal proposto no início do projeto, as
expectativas puderam ser cumpridas, tendo em vista os pontos citados no
parágrafo anterior. Mesmo com todos os problemas e desafios que foram
enfrentados e foram relatados, principalmente pelo capítulo anterior, mas
também ao longo de todos os outros, onde pode se destacar todos os empecilhos
relacionados com a raspagem de dados a partir de redes sociais, foi possível
realizar execuções completas de todo o processo automático descrito na seção de
fluxo da plataforma em algumas ocasiões, começando desde a busca de postagens feitas nas
páginas cadastradas, passando pela etapa de extração das informações a partir do processamento textual, até a criação e exibição do evento dentro da plataforma.

É viável de se afirmar que o projeto como um todo teve de grande importância para se aplicar conhecimentos obtidos durante toda a graduação e que também são de grande utilidade e frequentemente utilizadas durante o dia a dia de um profissional na área de desenvolvimento de software, já que, ao longo do trabalho, assim como destacado durante este documento, as intenções de se manterem boas práticas de desenvolvimento foram mantidas tanto no caráter técnico, com código limpo, testes unitários e uma arquitetura simples e concisa, quanto no aspecto não técnico, envolvendo aplicações das metodologias ágeis e alinhamentos entre os membros sobre as prioridades e os pontos mais importantes a serem desenvolvidos.

Assim como relatado no capítulo anterior, existem pontos dentro do projeto que
podem ser melhorados e que poderiam ser explorados durante um
desenvolvimento futuro, de forma que o sistema pudesse adquirir um caráter bem
mais genérico, abrangendo outros tipos de regiões, ou aperfeiçoando-o até um
ponto que possa ser usado por outras instituições públicas como prefeituras, por exemplo.
Além disso, novas abordagens poderiam ser exploradas para contornar a grande
dificuldade de se trabalhar com raspagem de dados e toda a parte de
dependências de sistemas terceiros à plataforma.
