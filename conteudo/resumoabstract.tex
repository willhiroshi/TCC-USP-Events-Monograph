%!TeX root=../tese.tex
%("dica" para o editor de texto: este arquivo é parte de um documento maior)
% para saber mais: https://tex.stackexchange.com/q/78101

% As palavras-chave são obrigatórias, em português e em inglês, e devem ser
% definidas antes do resumo/abstract. Acrescente quantas forem necessárias.
\palavrachave{Desenvolvimento \textit{web}}
\palavrachave{\textit{Webscraping}}
\palavrachave{Eventos}

\keyword{Web development}
\keyword{Webscraping}
\keyword{Events}

% O resumo é obrigatório, em português e inglês. Estes comandos também
% geram automaticamente a referência para o próprio documento, conforme
% as normas sugeridas da USP.
\resumo{
Em uma universidade como a USP, existem várias atividades e eventos que ocorrem diariamente, mas às vezes elas acabam passando despercebidas ou não há um modo fácil de filtrar aquelas que gostaríamos de participar. Portanto, o objetivo deste trabalho é criar um sistema capaz de compilar eventos universitários publicados em redes sociais, e mostrá-los em uma plataforma \textit{web} acessível aos alunos. O esforço foi dividido em três diferentes frentes para a construção do mesmo: módulo de extração e processamento textual de postagens em redes sociais, \textit{front-end} desenvolvido no \textit{framework} React.JS para visualização de eventos e \textit{back-end} utilizando o \textit{framework} Django para gerenciamento de dados. O sistema final cumpre o que foi idealizado, permitindo que os alunos tenham mais facilidade para visualizar eventos de seu interesse dentro da faculdade. 
}

\abstract{
In a university like USP, there are various activities and events happening daily, however, sometimes they go unnoticed or there is no easy way to filter those we would like to participate in. Therefore, the goal of this project is to create a system capable of compiling university events published on social networks and displaying them on a web platform accessible to students. The effort during the development was divided into three different sections for its construction: a module for extracting and processing textual content from social media posts, a front-end developed in the React.JS framework for event visualization, and a back-end using the Django framework for data management. The final system fulfills what was envisioned, allowing students to easily view events of interest within the university.
}
