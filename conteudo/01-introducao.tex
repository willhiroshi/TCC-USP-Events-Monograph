%!TeX root=../tese.tex
%("dica" para o editor de texto: este arquivo é parte de um documento maior)
% para saber mais: https://tex.stackexchange.com/q/78101

%% ------------------------------------------------------------------------- %%

% "\chapter" cria um capítulo com número e o coloca no sumário; "\chapter*"
% cria um capítulo sem número e não o coloca no sumário. A introdução não
% deve ser numerada, mas deve aparecer no sumário. Por conta disso, este
% modelo define o comando "\unnumberedchapter".
\chapter{Introdução}
\label{cap:introducao}

\enlargethispage{.5\baselineskip}

A diversidade de eventos e atividades disponíveis em uma instituição de renome
como a \ac{USP} é notável. Desde palestras e conferências de destaque até
eventos esportivos, culturais e acadêmicos, os estudantes têm acesso a uma
infinidade de oportunidades para aprimorar seus conhecimentos e se envolverem
em atividades enriquecedoras.

A participação ativa dos alunos em eventos acadêmicos e extracurriculares é
fundamental para enriquecer a experiência universitária e contribuir para o
desenvolvimento pessoal e profissional. Como ressaltado por ~\citet{tinto:12}
em \emph{Leaving College: Rethinking the Causes and Cures of Student Attrition}
a participação em atividades fora da sala de aula está diretamente relacionada
à retenção estudantil e ao sucesso acadêmico. Além disso, estudos como o de
~\citet{astin:97} em \emph{What Matters in College: Four Critical Years
    Revisited} destacam que o envolvimento dos alunos em atividades
extracurriculares está positivamente associado ao desenvolvimento de
habilidades de liderança, resolução de problemas e ao aumento da satisfação com
a experiência universitária.

Atualmente, as redes sociais desempenham um grande papel no acesso à
informação, proporcionando fácil acesso a diversos \textit{sites} e páginas por
meio de \textit{smartphones} ou computadores. Porém, justamente pela abundância
de diferentes fontes de informação, novos alunos podem se sentir perdidos
frente a este mar de oportunidades e tantas opções de atividades disponíveis.
Nesse sentido, encontrar informações e divulgações de eventos verdadeiramente
relevantes pode se tornar desafiador.

Assim, surge a necessidade de um sistema que possa simplificar o acesso a esses
dados. Uma plataforma simples e intuitiva que seja capaz de juntar essas
informações, a partir de diferentes publicações de redes sociais, e que permita
que seus usuários sejam capazes de visualizar, de forma clara e concisa,
informações sobre localização, horário e preço dos eventos que sejam de seu
interesse.

Neste contexto, este Trabalho de Conclusão de Curso aborda a criação de um
sistema que seja capaz de atender esta demanda e, consequentemente, facilitar e
incentivar a participação de alunos na vida universitária. É importante
ressaltar que este projeto tem como foco inicial apenas o campus da
Cidade Universitária da USP, mas que possa ser escalável para outras
universidades e até mesmo regiões específicas.